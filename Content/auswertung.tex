% !TeX root = ../my-thesis.tex
\chapter{Auswertung}

\section{Zusammenfassung}

Mobile Geräte erfreuen sich einer stetig wachsenden Beliebtheit in fast allen Lebensbereichen und sind mittlerweile in vielen Branchen unverzichtbare Arbeitswerkzeuge. Leider ist die permanente Nutzung dieser nützlichen Hilfsmittel durch die Akkukapazität zeitlich begrenzt. Ziel dieser Arbeit war es daher, Implementierungen von Algorithmen möglichst  energieeffizient umzusetzen. Dabei lagen die Schwerpunkt zum einen auf der Untersuchung paralleler Berechnungen mithilfe von Multithreading und zum anderen auf der Gegenüberstellung rekursiver und iterativer Ausführungen hinsichtlich des Energieverbrauchs.

Um das Verhalten des Energieverbrauchs in Abhängigkeit der Thread-Anzahl und des gewählten Verfahrens zu messen, wurde eine Android-Applikation entwickelt. Hier wurde ein paralleler Base64-Decoder mit dynamisch wählbarer Thread-Anzahl implementiert, um die Abhängigkeit zwischen Energieverbrauch und Multithreading zu untersuchen. Des Weiteren wurde der Mergesort-Algorithmus in verschieden Ausführungen in die App integriert, da dieser sich sehr gut für eine Gegenüberstellung von iterativer und rekursiver Ausführung eignet. Es ist anzumerken, dass die hier genutzten Algorithmen durchaus austauschbar sind, da es hauptsächlich um die Untersuchung der Konzepte Multithreading und Rekursion beziehungsweise Iteration ging. Beispielsweise hätte die Untersuchung des Multithreadings auch mit anderen parallelisierbaren Algorithmen durchgeführt werden können. Mithilfe des Programms \emph{Battery Historian} wurden die Android-Logdateien hinsichtlich der Spannung und des Entladestroms ausgelesen und anschließend grafisch ausgewertet. Aus den so ermittelten Messwerten und Diagrammen konnten einige Erkenntnisse bezüglich der Zielstellung gewonnen werden.

Es wurde festgestellt, dass mit steigender Thread-Anzahl bei der parallelen Ausführung nicht nur die Laufzeit sondern auch der Energieverbrauch solange sinkt, bis eine optimale Anzahl an Threads erreicht wurde. Bei weiterer Erhöhung der Thread-Anzahl steigen Energiebedarf und Laufzeit wieder an. Die optimale Anzahl an Threads ist hierbei abhängig von der vorliegenden Prozessorarchitektur. Im Fall des hier verwendeten Samsung Galaxy A7 mit seinen unterschiedlich starken Rechenkernen übersteigt die optimale Thread-Anzahl leicht die Anzahl der physischen Kerne. Für Geräte mit ausschließlich gleichstarken Rechenkernen ist jedoch eine Thread-Anzahl zu empfehlen, die der Anzahl an physischen Kernen entspricht.

Weiterhin konnte gezeigt werden, dass iterative Implementierungen energiesparender als rekursive Lösungen sind, sofern die zugrundeliegenden Algorithmen in der selben Komplexitätsklasse liegen. Da die Auslastung der \ac{cpu} für beide Ansätze nahezu identisch ausfällt, ist der erhöhte Energieaufwand rekursiver Verfahren mit der höheren Speicherbelastung zu begründen. Häufige Speicherzugriffe sind charakteristisch für rekursive Verfahren und führen zu erhöhten Laufzeiten. Vor allem für mobile Geräte ist ein speicherschonendes Vorgehen angebracht, da der vergleichsweise kleine Arbeitsspeicher neben  der Akkukapazität ein weiterer begrenzender Faktor ist. So kann eine aufwendige Rekursion eine mögliche Ursache für häufig auftretenden \ac{oom}-Abstürze sein.



\section{Ausblick}

Das Thema der energieeffizienten Implementierung auf mobilen Geräten bietet auch für künftige Arbeiten und Untersuchungen weitere Ansätze mit anderen Schwerpunkten. Da der Einfluss der Prozessorarchitektur im Rahmen der hier vorgenommen Messung eindeutig zutage kam, ist es sinnvoll diesen Aspekt zukünftig näher zu beleuchten. So könnten verschiedene Prozessorarchitekturen mobiler Geräte hinsichtlich der Energieeffizienz verglichen werden. Darüber hinaus könnte für eine spezifische Architektur, die optimale Implementierung für parallele Ausführung ermittelt werden. Einen weiteren Ansatz bietet die Betrachtung der Speichereffizienz von verschiedenen Algorithmen und Vorgehensweisen, da Speicherzugriffe signifikanten Einfluss auf die Laufzeit haben und dadurch auch den Energieverbrauch beeinflussen.
