% !TeX root = ../my-thesis.tex
\chapter{Einleitung}
\section{Thematische Einführung}
Mobile Gräte wie Smartphones oder Tablets sind aus dem heutigen Alltag der Menschen nicht mehr wegzudenken. Ob zur privaten Unterhaltung beim Spielen aufwendiger 3D-Spiele, beim Streamen von Musik- und Videoinhalten oder als unentbehrliches Werkzeug bei der täglichen Arbeit, mobile Geräte sind nahezu den gesamten Tag über im Einsatz. Auch die Außendiensttechniker der DT Technik GmbH nutzen mobile Applikationen wie die Bestell-App oder die Mess-App zur Verwaltung und Aufstockung ihrer Werkzeuge und Ersatzteile beziehungsweise zur Untersuchung und Konfiguration von Routern. Hohe Akkulaufzeiten sind für diese Art der Benutzung eine Voraussetzung und stellen Smartphone Hersteller sowie Softwareentwickler vor die Herausforderung, energieeffiziente Lösungen zu finden.
Dabei werden verschiedene Ansätze verfolgt. So versuchen die Hersteller energiesparende Prozessoren und Displays zu entwickeln und die Hardwarenutzung zu optimieren. Dieses Bestreben steht jedoch im Konflikt mit den Wünschen der Kunden, welche schnellere Mehrkernprozessoren und größere Displays für die neuen Geräte erwarten. Dieser Trend ist auch in der Marktentwicklung der letzten zehn Jahre zu beobachten. So entsprach die durchschnittliche Displaygröße 2009 c.a. 3,2 Zoll. Acht Jahre später waren bereits Displays mit 5,5 Zoll üblich \cite{DisplayGroesse}. Auf der anderen Seite versuchen Anwendungsentwickler durch Softwareoptimierung ihre Applikationen ressourcensparender zu gestalten. Dabei können Prozesse wie zum Beispiel größere Downloads für Datenbankupdates als Service oder mithilfe des neuen Android Work Managers als Hintergrundprozess implementiert werden und in Abhängigkeit vom aktuellen Ladestand des Akkus auf günstigere Zeitpunkte verschoben werden \cite{WorkManager}. Auch das gewissenhafte Umgehen mit Wake Locks und das Festlegen der Standby Phasen der eigenen Applikation sind wichtige Stellschrauben, über welche ein Anwendungsentwickler energiesparende Anpassungen justieren kann \cite{WakeLocks}. Dies sind nur einige Beispiele der möglichen Optionen für Energieoptimierungen auf mobilen Geräten. Das Problem der Energieeffizienz auf mobilen Geräten bietet ein breites Feld an Forschungspotential und wird auch zukünftig eine zentrale Rolle in der Geräteherstellung und App-Entwicklung spielen.

\section{Thematische Abgrenzung und Zielstellung dieser Arbeit}

In dieser Arbeit werden verschiedene Implementierungsansätze für Algorithmen und Berechnungen betrachtet und auf deren Einfluss hinsichtlich des Energieverbrauchs mobiler Geräte untersucht. Der Schwerpunkt liegt hierbei auf der Betrachtung von parallelen Berechnungen mithilfe von Multithreading. Dabei wird der Zusammenhang zwischen der Anzahl der parallel laufenden Threads, der Laufzeitveränderung und dem damit einhergehenden Energieverbrauch gemessen.
Ziel dieser Untersuchung ist es, herauszufinden, ob es einen Zusammenhang zwischen den drei genannten Parametern gibt und gegebenenfalls eine Empfehlung für die Implementierung von Multithreading zu konstruieren, welche einen sinnvollen Kompromiss aus Laufzeit und Energieverbrauch bereitstellt.
Weiterhin wird der Unterschied zwischen rekursiven und iterativen Implementierungen hinsichtlich des Stromverbrauchs betrachtet. Auch hier ist das Ziel, die ressourcenschonendste Variante zu ermitteln.


\section{Struktureller Aufbau und Vorgehensweise der Untersuchung}

Das anschließende Kapitel \glqq Untersuchte Konzepte und Implementierungen\grqq{} beschreibt die theoretischen Aspekte von paralleler Programmierung und deren Einfluss auf den Energieverbrauch sowie die für diese Arbeit relevanten Gesetzmäßigkeiten und Berechnungsformeln des Multiprocessings. Außerdem werden die zur Untersuchung entwickelten Implementierungen vorgestellt, welche in der \glqq EnergyEfficience\grqq{} Applikation angewandt werden. Für die Umsetzung einer parallelen Ausführung wurde ein Base64 Encoder implementiert. Dieser ist nur ein Mittel zum Zweck und könnte problemlos durch andere parallelisierbare Algorithmen ersetzt werden. Des Weiteren werden auch die Unterschiede von rekursiven und iterativen Algorithmen beleuchtet und gegenübergestellt. Diese Gegenüberstellung bildet den zweiten große Untersuchungsgegenstand dieser Arbeit. Weil der Mergesort Algorithmus aufgrund seiner Vielseitigkeit perfekt für den Vergleich dieser beiden Implementierungsstrategien geeignet ist, werden in diesem Kapitel die verwendeten Mergesort Varianten inklusive ihrer Implementierung vorgestellt.

Im darauffolgenden Kapitel drei \glqq Verwendete Geräte und Tools\grqq{} werden alle Programme und Geräte samt deren Spezifikationen vorgestellt, die im Rahmen dieser Untersuchung genutzt wurden. Es wird eine kurze Einführung in die Verwendung des Programms Battery Historian geben. Weiterhin wird die eigens für diese Arbeit entwickelte App vorgestellt. Anschließend wird die Messmethode mithilfe dieser beiden Tools beschrieben.

In Kapitel vier \glqq Energieeffizientes Multithreading\grqq{} werden die Ergebnisse der Messungen des parallelen Base64 Encoders dargestellt und ausgewertet. Der Fokus liegt hierbei auf dem Zusammenhang zwischen verwendeter Thread-Anzahl und dem Energieverbrauch.

Daraufhin untersucht Kapitel fünf \glqq Rekursive und Iterative Verfahren im Vergleich\grqq{} den Einfluss von iterativer beziehungsweise rekursiver Implementierung auf den Energieverbrauch. Hierfür wird die bereits erwähnte Mergesort Implementierung der \glqq EnergyEfficience\grqq{} Applikation genutzt. Außerdem werden die Vorzüge des Fork-Join Thread Pools bei paralleler Rekursion durch Messungen untersucht..

Im abschließenden Kapitel sechs \glqq Auswertung\grqq{} ist eine Zusammenfassung der gewonnen Erkenntnisse zu finden. Außerdem werden in einem Ausblick Ansätze für weitere Untersuchungen auf diesem Gebiet formuliert.

