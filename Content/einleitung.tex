% !TeX root = ../my-thesis.tex
\chapter{Einleitung}
\section{Thematische Einführung}
Mobile Gräte wie Smartphones oder Tablets sind aus dem heutigen Alltag der Menschen nicht mehr wegzudenken. Ob zur privaten Unterhaltung beim spielen aufwendiger 3D Spiele, streamen von Musik- und Videoinhalten oder als unentbehrliches Werkzeug bei der täglichen Arbeit, mobile Geräte sind nahezu den gesamten Tag im Einsatz. Auch die Außendiensttechniker der DT Technik GmbH nutzen mobile Applikationen wie die Bestell-App oder die Mess-App zur Verwaltung und Aufstockung ihrer Werkzeuge und Ersatzteile beziehungsweise zur Untersuchung und Konfiguration von Routern. Hohe Akkulaufzeiten sind für diese Art der Benutzung eine Voraussetzung und stellen Smartphone Hersteller sowie Softwareentwickler vor die Herausforderung energieeffiziente Lösungen zu finden.
Dabei werden verschiedene Ansätze verfolgt. Auf der einen Seite versuchen die Hersteller energiesparende Prozessoren und Displays zu entwickeln und die Hardwarenutzung zu optimieren. Dieses Bestreben steht jedoch im Konflikt mit den Wünschen der Kunden, welche schnellere Mehrkern Prozessoren und größere Displays für die neuen Geräte erwarten. Dieser Trend ist auch in der Marktentwicklung der letzten zehn Jahre zu beobachten. So entsprach die durchschnittliche Displaygröße 2009 c.a. 3,2 Zoll. Acht Jahre später waren bereits Displays mit 5,5 Zoll üblich.\todo{Quellen Displaygröße} Auf der anderen Seite versuchen Anwendungsentwickler durch Softwareoptimierung ihre Applikationen ressourcensparender zu gestalten. Dabei können Prozesse wie zum Beispiel größere Downloads für Datenbankupdates als Service oder mithilfe des neuen Android Work Manager's als Hintergrundprozess implementiert werden und in Abhängigkeit vom aktuellen Ladestand des Akkus auf günstigere Zeitpunkte verschoben werden.\todo{Quelle WorkManager} Auch das gewissenhafte Umgehen mit Wake Locks und das festlegen der Standby Phasen der eigenen Applikation sind wichtige Stellschrauben, über welche ein Anwendungsentwickler energiesparende Anpassungen justieren kann.\todo{Quelle Appstandby und Wake locks} Dies sind nur einige Beispiele  der möglichen Optionen für Energieoptimierungen auf mobilen Geräten. Das Problem der Energieeffizienz auf mobilen Geräten bietet ein breites Feld an Forschungspotential und wird auch zukünftig eine zentrale Rolle in der Geräteherstellung und App-Entwicklung spielen.
\section{Thematische Abgrenzung und Zielstellung dieser Arbeit}
In dieser Arbeit werden verschiedene Implementierungsansätze für Algorithmen und Berechnungen betrachtet und auf deren Einfluss auf den Energieverbrauch mobiler Geräte untersucht. Der Schwerpunkt liegt hierbei auf der Betrachtung von parallelen Berechnungen mithilfe von Multithreading. Dabei wird der Zusammenhang zwischen der Anzahl der parallel laufenden Threads, der Laufzeitveränderung und dem damit einhergehenden Energieverbrauch gemessen.
Ziel dieser Untersuchung ist es, herauszufinden ob es einen Zusammenhang zwischen den drei genannten Parametern gibt und gegebenenfalls eine Empfehlung für die Implementierung von Multithreading herauszuarbeiten, welche einen sinnvollen Kompromiss aus Laufzeit und Energieverbrauch bereitstellt.
Weiterhin wird der Unterschied zwischen rekursiven und iterativen Implementierungen hinsichtlich des Stromverbrauchs betrachtet. Auch hier ist das Ziel, die ressourcenschonendste Variante zu ermitteln.


\section{Struktureller Aufbau und Vorgehensweise der Untersuchung}

Im folgenden Kapitel 2 \glqq Verwendete Geräte und Tools\grqq{} werden alle Programme und Geräte samt deren Spezifikationen vorgestellt, die im Rahmen dieser Untersuchung genutzt wurden. Es wird eine kurze Einführung in die Verwendung des Programms Battery Historian geben. Weiterhin wird die eigens für diese Arbeit entwickelte App vorgestellt. Anschließend wird die Messmethode mithilfe dieser beiden Tools beschrieben.
Kapitel 3 \glqq Energiesparen auf Smartphones Stand der Technik\grqq{} beschreibt aktuelle Maßnahmen in der Hardware- und Androidentwicklung um dem Problem der begrenzten Akkulaufzeit zu begegnen.\todo{vlt Kapitel ersetzen}
In Kapitel 4 \glqq Energieeffizientes Multithreading\grqq{} folgt die Beschreibung theoretischer Grundlagen des Multithreadings auf Androidgeräten. Für die Durchführung der Messung wurde ein Base64 Encoder implementiert, welcher hinsichtlich Laufzeitkomplexität erklärt wird. Außerdem werden die Vorzüge und die Implementierung des verwendeten Thread Pool Managers erläutert.
Der Hauptbestandteil des Kapitels besteht aus der Darstellung und Auswertung der Messwerte des Versuchs.
Das anschließende Kapitel 5 \glqq Rekursive und Iterative Verfahren im Vergleich\grqq{} beinhaltet die Untersuchung von iterativen und rekursiven Mergesort Implementierungen. Zu Beginn werden Iteration und Rekursion auf kostentechnische Aspekte verglichen, die drei verschiedenen Mergesort Varianten werden beschrieben und die Vorzüge des verwendeten Fork-Join-Thread Pools werden erläutert. Anschließend folgt die Darstellung und Auswertung der Messwerte.
Im abschließenden Kapitel 6 \glqq Auswertung\grqq{} ist eine Zusammenfassung der gewonnen Erkenntnisse zu finden. Außerdem wird eine Empfehlung ausgesprochen, welche auf diesen Erkenntnissen beruht.

